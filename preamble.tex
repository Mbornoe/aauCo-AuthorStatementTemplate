%  A simple template for Co-author statements in relation to submitting PhD Thesis.
%  2018-09-08 v. 1.0.0
%  Copyright 2018 by Morten Bornø Jensen <mboj@create.aau.dk>
%
%  This is free software: you can redistribute it and/or modify
%  it under the terms of the GNU General Public License as published by
%  the Free Software Foundation, either version 3 of the License, or
%  (at your option) any later version.
%
%  This is distributed in the hope that it will be useful,
%  but WITHOUT ANY WARRANTY; without even the implied warranty of
%  MERCHANTABILITY or FITNESS FOR A PARTICULAR PURPOSE.  See the
%  GNU General Public License for more details.
%
%  You can find the GNU General Public License at <http://www.gnu.org/licenses/>.
%
\documentclass[12pt,oneside]{report}

\usepackage[top=1in, bottom=1.25in, left=1.0in, right=1.0in]{geometry}


%%%%%%%%%%%%%%%%%%%%%%%%%%%%%%%%%%%%%%%%%%%%%%%%
% Language, Encoding and Fonts
% http://en.wikibooks.org/wiki/LaTeX/Internationalization
%%%%%%%%%%%%%%%%%%%%%%%%%%%%%%%%%%%%%%%%%%%%%%%%
% Select encoding of your inputs. Depends on
% your operating system and its default input
% encoding. Typically, you should use
%   Linux  : utf8 (most modern Linux distributions)
%            latin1 
%   Windows: ansinew
%            latin1 (works in most cases)
%   Mac    : applemac
% Notice that you can manually change the input
% encoding of your files by selecting "save as"
% an select the desired input encoding. 
\usepackage[utf8]{inputenc}
\usepackage[danish,english]{babel}
\usepackage{ragged2e}

% Choose the font encoding
\usepackage[T1]{fontenc}


\usepackage{graphicx}
\usepackage{tikz}


% load a colour package
\usepackage{xcolor}
% UniPrint prefers that no colours are used in the template by default.
% However, if you want some colours, then you can easily change the following
%\definecolor{aaublue}{RGB}{0,0,0}% black
\definecolor{aaublue}{RGB}{33,26,82}% dark blue

%%%%%%%%%%%%%%%%%%%%%%%%%%%%%%%%%%%%%%%%%%%%%%%%
% Page Layout and appearance
% http://en.wikibooks.org/wiki/LaTeX/Page_Layout
%%%%%%%%%%%%%%%%%%%%%%%%%%%%%%%%%%%%%%%%%%%%%%%%
%\usepackage[
%  outer=2.0cm, % right margin on an odd page
%  inner=2.0cm, % left margin on an odd page
%  top=2.0cm, % top margin
%  bottom=2.0cm % bottom margin
%  ]{geometry}% Change margins, papersize, etc of the document

\pagestyle{empty} 




\makeatletter
\newcommand{\phdSudent}[1]{\newcommand{\@phdStudent}{#1}}
\newcommand{\thePhdStudent}[0]{\@phdStudent}
\makeatother

% create the paper titlepage
\newcommand{\papertitlepage}[2]{
  {\raggedright \Large \bfseries \color{aaublue}#1 \par} \vskip 1em
  {\raggedright \footnotesize #2 \par}
  \vskip -0.8em
  \noindent\makebox[\linewidth]{\rule{\textwidth}{0.6pt}}
}

\newcommand{\paperInfo}[4]{
  {\justify \small \bfseries Paper Title: \mdseries \itshape #1 \par} \vskip 1em
%  {\justify \small \bfseries Paper number in the thesis: \mdseries \itshape #2 \par} \vskip 1em
  {\justify \small \bfseries Publication outlet: \mdseries \itshape #2 \par} \vskip 1em
  {\justify \small \bfseries List of authors: \mdseries \itshape #3 \par} \vskip 1em
  {\justify \small \bfseries PhD Student: \mdseries \itshape \thePhdStudent \par} \vskip 1em
  {\justify \small \bfseries Scientific contribution of the PhD student (all participating PhD students) to the paper: \mdseries \itshape #4 \par} \vskip 1em
}

\newcommand{\authorSignature}[1]{
	{\raggedright \small #1
}	\vskip -0.8em
    \noindent\makebox[\linewidth]{\rule{\textwidth}{0.6pt}}
}

\def\stack{}
\def\stackcount{0}

\def\author#1{{%
#1\ignorespaces
\let\xdo\relax
\count0=\stackcount\relax
\advance\count0 by 1
\xdef\stackcount{\the\count0}%
\toks0{\xdo{#1}}%
\toks2\expandafter{\stack}%
\xdef\stack{\the\toks2 \the\toks0 }}}


\def\printAuthorSignatures{{\count0=0
\def\xdo##1{\advance\count0 by 1
##1 \vskip -0.8em \noindent\makebox[\linewidth]{\rule{\textwidth}{0.6pt}} \vskip 2em }%
\stack}}


\makeatletter
\def\ifshowTemplate{\ifdim\overfullrule>\z@
  \expandafter\@firstoftwo\else\expandafter\@secondoftwo\fi}
\makeatother


\newcommand{\statement}[2]{
	\justify
	\begin{itemize}
	\item[•] \textbf{#1} #2
	\end{itemize}
}
